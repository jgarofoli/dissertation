% su style guidelines
% http://www.syr.edu/gradschool/em/pdfs/emdissertation/Doctoral%20Dissertation%20%20Masters%20theses%20Format%20Guidelines.pdf
% https://en.wikipedia.org/wiki/Book_design

%\usepackage[T1]{fontenc}

\usepackage{ifthen}
\newboolean{uprightparticles}
\setboolean{uprightparticles}{false} %True for upright particle symbols

\newboolean{isdraft}
\setboolean{isdraft}{false} % change to false when final version is ready

\usepackage{xspace}
\usepackage{adjustbox} % for clipping seal images in xelatex
\usepackage{sidecap}
\usepackage{rotating} % for sidewaysfigure

\usepackage[english]{babel}
\usepackage[mdyy]{datetime}
\usepackage[margin=1in]{geometry}
\usepackage{everysel}
\usepackage{amsmath,amssymb}
%\usepackage[nomarkers]{endfloat}
%\usepackage{bm}
\usepackage{bookmark,hyperref}
\usepackage[all]{hypcap} 
%\usepackage{feynmf}
\usepackage[nottoc]{tocbibind}
\usepackage{appendix}
\usepackage{color}
\usepackage{tabu}
\usepackage{xfrac}

\usepackage{tikz}
\usetikzlibrary{arrows,decorations.pathmorphing,backgrounds,positioning,fit,petri,decorations.markings,patterns,plotmarks,shapes}

% if using xelatex, will also format math
%\usepackage[Numbers=OldStyle,Ligatures=TeX,Scale=MatchLowercase]{mathspec}
%\setallmainfonts{Source Sans Pro Light}
%\newfontfamily\semibold{Source Sans Pro}
%\newcommand\tracked[1]{{\addfontfeature{LetterSpace=12}#1}} % for all caps words
%\usepackage{xunicode}
%\defaultfontfeatures{Mapping=tex-text}
%\usepackage{unicode-math}
%\usepackage{hvmath}
%\usepackage[Euler]{upgreek}

\usepackage[default,light,oldstyle,semibold]{sourcesanspro}
\usepackage[cmbright]{sfmath}

\usepackage[tracking=true]{microtype}
\SetTracking
 [ no ligatures = {f},
 spacing = {600*,-100*, },
 outer spacing = {450,250,150},
 outer kerning = {*,*} ]
 { encoding = * }
 { 160 }
\newcommand\tracked[1]{\textls{#1}}


%\usepackage[default]{sourcesanspro} % doesn't change some any greek letters (only pi and o)
%\usepackage{microtype} % doesn't really work with xelatex
%\usepackage{sfmath}

\usepackage[font={it}]{caption}
\captionsetup[table]{aboveskip=12pt}
\captionsetup[table]{belowskip=12pt}



\usepackage[center]{titlesec}
\titleformat{\part}[display]{\Huge\filcenter\normalfont}{Part \thepart\\}{0pt}{\Huge\normalfont}
\titleformat{\chapter}[display]{\large\filcenter\normalfont}{\color{gray} Chapter \thechapter}{0pt}{\huge\normalfont}
\titleformat{\section}{\normalfont\filcenter\large}{\thesection}{1em}{}
\titleformat{\subsection}{\normalfont\filcenter\large}{\thesubsection}{1em}{}
\titleformat{\subsubsection}[runin]{\normalfont\scshape}{\thesubsubsection}{1em}{}[:]
\titleformat{\paragraph}[runin]{\normalfont\itshape}{\theparagraph}{}{}[:]
\titlespacing{\paragraph}{0em}{0em}{0.2em}
\usepackage{titletoc}

\usepackage{tocloft}
\renewcommand{\cftpartfont}{\Large \normalfont}
\renewcommand{\cftpartpagefont}{\large \normalfont}
\renewcommand{\cftchapfont}{\large \normalfont}
\renewcommand{\cftchappagefont}{\large \normalfont}
\renewcommand{\cfttoctitlefont}{\huge \normalfont}
\renewcommand{\cftloftitlefont}{\huge \normalfont}
\renewcommand{\cftlottitlefont}{\huge \normalfont}
%\makeatletter
%\renewcommand{\@numwidth}{2.55em}
%\makeatother

\frenchspacing
\usepackage{parskip}
\setlength{\parskip}{0.35cm}
\usepackage{setspace}
%\onehalfspace
\doublespace

\usepackage{sparklines}

\ifthenelse{\boolean{isdraft}}{
\usepackage{lineno}
\linenumbers
% there is a problem with amsmath and lineno stuff, so here is a fix
% from http://phaseportrait.blogspot.com/2007/08/lineno-and-amsmath-compatibility.html
\newcommand*\patchAmsMathEnvironmentForLineno[1]{%
  \expandafter\let\csname old#1\expandafter\endcsname\csname #1\endcsname
  \expandafter\let\csname oldend#1\expandafter\endcsname\csname end#1\endcsname
  \renewenvironment{#1}%
     {\linenomath\csname old#1\endcsname}%
     {\csname oldend#1\endcsname\endlinenomath}}% 
\newcommand*\patchBothAmsMathEnvironmentsForLineno[1]{%
  \patchAmsMathEnvironmentForLineno{#1}%
  \patchAmsMathEnvironmentForLineno{#1*}}%
\AtBeginDocument{%
\patchBothAmsMathEnvironmentsForLineno{equation}%
\patchBothAmsMathEnvironmentsForLineno{align}%
\patchBothAmsMathEnvironmentsForLineno{flalign}%
\patchBothAmsMathEnvironmentsForLineno{alignat}%
\patchBothAmsMathEnvironmentsForLineno{gather}%
\patchBothAmsMathEnvironmentsForLineno{multline}%
}
}{ % no line numbers for not a draft
    }

\usepackage{fancyhdr}

\setlength{\headheight}{15pt}
\fancypagestyle{plain}{%
\fancyhf{} % clear all header and footer fields
\ifthenelse{\boolean{isdraft}}{
\fancyhead[LO,RE]{\small - * - \tracked{DRAFT} \today~- * -}
}{}
\fancyhead[RO]{\small \tracked{\scriptsize \leftmark}\hspace{3mm}\thepage} % except the center
\renewcommand{\headrulewidth}{0pt}
\renewcommand{\footrulewidth}{0pt}
\renewcommand{\headheight}{16pt}
\fancyhead[LE]{\small \thepage\hspace{3mm}\tracked{\scriptsize \leftmark}} % except the center
\renewcommand{\headrulewidth}{0pt}
\renewcommand{\footrulewidth}{0pt}
\renewcommand{\headheight}{16pt}}

\fancypagestyle{plainbottom}{%
\fancyhf{} % clear all header and footer fields
\ifthenelse{\boolean{isdraft}}{
\fancyhead[LO,RE]{\small - * - \tracked{DRAFT \today}~- * -}
}{}
\fancyfoot[C]{\small \thepage} % except the center
\renewcommand{\headrulewidth}{0pt}
\renewcommand{\footrulewidth}{0pt} }

\fancypagestyle{empty}{%  just delete this when not in draft any more
\fancyhf{} % clear all header and footer fields
\ifthenelse{\boolean{isdraft}}{
\fancyhead[LO,RE]{\small - * - \tracked{DRAFT \today}~- * -}
}{}
\renewcommand{\headrulewidth}{0pt}
\renewcommand{\footrulewidth}{0pt}}

\pagestyle{plainbottom}





%% from http://tex.stackexchange.com/questions/22100/the-bar-and-overline-commands
\makeatletter
\newsavebox\myboxA
\newsavebox\myboxB
\newlength\mylenA
\newcommand*\xoverline[2][0.75]{%
    \sbox{\myboxA}{$\m@th#2$}%
    \setbox\myboxB\null% Phantom box
    \ht\myboxB=\ht\myboxA%
    \dp\myboxB=\dp\myboxA%
    \wd\myboxB=#1\wd\myboxA% Scale phantom
    \sbox\myboxB{$\m@th\overline{\copy\myboxB}$}%  Overlined phantom
    \setlength\mylenA{\the\wd\myboxA}%   calc width diff
    \addtolength\mylenA{-\the\wd\myboxB}%
    \ifdim\wd\myboxB<\wd\myboxA%
       \rlap{\hskip 0.5\mylenA\usebox\myboxB}{\usebox\myboxA}%
    \else
        \hskip -0.5\mylenA\rlap{\usebox\myboxA}{\hskip 0.5\mylenA\usebox\myboxB}%
    \fi}
\makeatother

%% from http://tex.stackexchange.com/questions/109576/change-sqrt-symbol 
%\def\radice#1{\surd\kern-.25pt\overline{\mathstrut#1}}

%\makeatletter
%\let\ps@plain\ps@fancy
%\makeatother
\AtBeginDocument{\addtocontents{toc}{\protect\thispagestyle{plainbottom}}}
\AtBeginDocument{\addtocontents{lof}{\protect\thispagestyle{plainbottom}}}
\AtBeginDocument{\addtocontents{lot}{\protect\thispagestyle{plainbottom}}}

\DeclareTextCommandDefault{\nobreakspace}{\leavevmode\nobreak\ }

\DeclareGraphicsRule{*}{mps}{*}{}

\usepackage{stmaryrd}
\renewcommand\to{\rightarrowtriangle}
\renewcommand\rightarrow{\rightarrowtriangle}

\newcommand{\avg}[1]{\left< #1 \right>} % for average

\renewcommand\labelitemi{\small$\square$}

%\addto\captionsenglish{\renewcommand{\bibname}{References}}

\usepackage[hyperref=true,
            url=true,
            isbn=false,
            backref=true,
            style=numeric-comp,
            maxcitenames=3,
            maxbibnames=100,
            sorting=none,
            arxiv=abs,
            block=none]{biblatex}
\addbibresource{thesis.bib}
\DefineBibliographyStrings{english}{%
  bibliography = {References},
}

\renewcommand{\bibsetup}{\pagestyle{plain}}
